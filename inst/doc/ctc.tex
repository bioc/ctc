% -*- mode: noweb; noweb-default-code-mode: R-mode; -*-
% building this document: (in R) Sweave ("ctc.Rnw")
\documentclass[a4paper]{article}

\title{Ctc Package}
\author{Antoine Lucas}


%\usepackage{a4wide}
%\VignetteIndexEntry{Introduction to ctc}
%\VignettePackage{ctc}

\usepackage{/usr/local/R/R-dev_jul05/lib/R/share/texmf/Sweave}
\begin{document}

\maketitle

\tableofcontents

\section{Overview}

{\tt Ctc} package provides several functions for 
conversion. Specially to export and import data from
Xcluster\footnote{http://genome-www.stanford.edu/\~\/sherlock/cluster.html} 
 or Cluster\footnote{http://rana.lbl.gov/EisenSoftware.htm} 
 software (very used for Gene's expression
analysis).

\section{Aim} 

\begin{itemize}
\item To explore clusters made by Xcluster and Cluster . 

\item To cluster data with Xcluster (it requires very low memory usage) 
and analyze the results with R. Warning: results are not exactly the same
as hclust results with R. 
\end{itemize}              
       
\section{Usage}

Standard way of building a hierarchical clustering with R
is with this command:
%<<echo=TRUE,fig=TRUE>>=
\begin{Schunk}
\begin{Sinput}
> data(USArrests)
> h = hclust(dist(USArrests))
> plot(h)
\end{Sinput}
\end{Schunk}
Or for the ``heatmap'':
\begin{Schunk}
\begin{Sinput}
> heatmap(as.matrix(USArrests))
\end{Sinput}
\end{Schunk}
\includegraphics{ctc-002}


\subsection{Building hierarchical clustering with another software}

We made these tools
\begin{description}
\item[r2xcluster] Write data table to Xcluster file format 
\begin{Schunk}
\begin{Sinput}
> library(ctc)
> r2xcluster(USArrests, file = "USArrests_xcluster.txt")
\end{Sinput}
\end{Schunk}
 \item[r2cluster] Write data table to Cluster file format 
\begin{Schunk}
\begin{Sinput}
> r2cluster(USArrests, file = "USArrests_xcluster.txt")
\end{Sinput}
\end{Schunk}
\item[xcluster] Hierarchical clustering (need Xcluster tool  by Gavin Sherlock) 
\begin{verbatim}
> h.xcl=xcluster(USArrests)
> plot(h.xcl)
\end{verbatim}
 
It is roughtly the same as
\begin{verbatim}
> r2xcluster(USArrests,file='USArrests_xcluster.txt')
> system('Xcluster -f USArrests_xcluster.txt -e 0 -p 0 -s 0 -l 0')
> h.xcl=xcluster2r('USArrests_xcluster.gtr',labels=TRUE)
\end{verbatim}


\item[xcluster2r] Importing Xcluster/Cluster output 

\end{description}                 

\subsection{Using other visualization softwares}

We now consider that we have an object of the type produced by 'hclust'
(or a hierarchical cluster imported with previous functions) like:

\begin{Schunk}
\begin{Sinput}
> hr = hclust(dist(USArrests))
> hc = hclust(dist(t(USArrests)))
\end{Sinput}
\end{Schunk}



\begin{description}
\item[hc2Newick] Export hclust objects to Newick format files
\begin{Schunk}
\begin{Sinput}
> write(hc2Newick(hr), file = "hclust.newick")
\end{Sinput}
\end{Schunk}
\item[r2gtr,r2atr,r2cdt] Export hclust objects to Freeview or Treeview
visualization softwares
\begin{Schunk}
\begin{Sinput}
> r2atr(hc, file = "cluster.atr")
> r2gtr(hr, file = "cluster.gtr")
> r2cdt(hr, hc, USArrests, file = "cluster.cdt")
\end{Sinput}
\end{Schunk}
\end{description}
    
\section{See Also}

Theses examples can be tested with command
{\tt demo(ctc)}.\\


\noindent
All functions has got man pages, try 
{\tt help.start()}.\\

\noindent
Ctc aims to interact with other softwares, some of them:
\begin{description}
\item[xcluster]
made  by Gavin Scherlock, http://genome-www.stanford.edu/\~\/sherlock/cluster.html
\item[Cluster, Treeview]
made  by Michael Eisen, http://rana.lbl.gov/EisenSoftware.htm
\item[Freeview]
made by Marco Kavcic and Blaz Zupan,
http://magix.fri.uni-lj.si/freeview
\end{description}                 


\end{document}


